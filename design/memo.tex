\documentclass[twocolumn,10pt]{jarticle}

\usepackage[dvipdfmx]{graphicx}

\graphicspath{{./image/}}

\title{実験計画}

\author{奥屋 直己}

\usepackage[height=26cm,width=16cm]{geometry}

\begin{document}

\maketitle

\section{メモ}
\section{28J00006393805.pdf}
4分音符から16分音符へ移り変わる時,どの速度が調度良く聞こえるかの実験\\
BPM150,100,75で実験した\\
極限法と調整法の2つの実験を行った\\
極限法では16分音符のところを変えてみてどれが「はやい」「おそい」「どちらでもない」を判断を求める\\
調整法は被験者に16分音符の調整を行ってもらい,正しいところまで持っていってもらう\\
\subsection{結果}
\section{分遣ないやつ}
Stevens, L. (1886). On the time-sense.
水戸さんとむらおさんのやつ
\end{document}
