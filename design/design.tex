 \documentclass{jsarticle}

\usepackage[dvipdfmx]{graphicx}

\title{実験計画}

\author{奥屋 直己}

\usepackage[height=26cm,width=16cm]{geometry}

\begin{document}

\maketitle

\section{目的}
音楽の演奏において、テンポを一定に保ち続けることは最も重要なことの一つである。しかし、演奏者の意図にかかわらずにテンポ変化が生じることがあり、そのような場合の多くは、テンポが加速する方向に変化する。このような意図しないテンポ変化は、演奏者の間では「走る」という表現で共有され、だれもがよく経験することである。このテンポ加速現象は古くから検討されてきたが、従来の実験の多くが一定時間間隔のタッピング課題を対称としており、リズムや強弱といった実際の音楽演奏に含まれる時間パタンの要素を考慮していない(2.永島 亮誠, 阪口 豊)。また、実際の音楽演奏において、例えば、打楽器演奏での振り上げる腕の高さや、叩く時の腕の速度など、演奏者の体の動きもテンポ加速現象に影響を及ぼすと考える。本実験ではリズムやアクセントパタンを伴うタッピングの同期課題と、タッピングする腕の位置を測定し、これらの要素がテンポ維持特性に与える影響を実験的に検討する。

\section{実験計画}
本実験には、従来の研究と同様に同期・継続課題を用いた。従来の研究と異なる点では、テンポだけでなくリズムを含めて目標音と同期してタッピングし、また、それを継続する点である。本実験では、同期区間の時間長を、一定間隔条件でのタッピング1拍分の時間を単位として32拍分、継続区間を320拍分とした。

本研究では、Collyer,et al. の報告[1]においてテンポ変化が生じにくかった120 bpmを目標テンポに設定して目標リズム音を作成した。
\subsection{手続き}
核実験での課題になれるために、本試験にのまえに、臓器区間8小節、継続区間8小節の練習課題を課す。被験者がリズムパタンを理解したことを確認した後、被験者がリズムパタンを理解したことを確認した後、同期区間 32拍分、継続区間 320拍分の本試験を行う。

\section{文献}
\begin{enumerate}
  \item Collyer, C., Broadbent, H., \& Church, R.(1992).Categorical time production: Evidence for discrete timing in motor control. {\it Preception \& Psychonomic bulletin \& Psychophysics, 51(2),134-144.}
  \item 永島 亮誠, 阪口 豊. (2018) "リズムパタンや強弱の時間パタンがテンポ維持特性に与える影響"
\end{enumerate}
\end{document}